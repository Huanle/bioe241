\documentclass{article}
\pagestyle{empty}

\begin{document}

\section*{Papers}

\begin{itemize}
\item Matsumoto {\em et al}, 2000. {\bf Biological sequence compression algorithms.}
% \alert{Finds repeats, palindromes; c.f. LZ77, LZ78}
\item Chen {\em et al}, 2002. {\bf DNACompress: fast and effective DNA sequence compression.}
% \alert{Finds repeats}
\item Christley {\em et al}, 2009. {\bf Human genomes as email attachments.}
% \alert{James Watson's genome in 4MB}
\item Baldi {\em et al}.
{\bf Data structures and compression algorithms for high-throughput sequencing technologies.}
BMC Bioinformatics, 2010.
\item Birney {\em et al}.
{\bf Efficient storage of high throughput sequencing data using reference-based compression.}
Genome Research, 2011.
\end{itemize}

\section*{Questions}

\begin{enumerate}
\item What are the stated applications of each tool? How broadly/narrowly focused are they?
\item What codes are used? What kind of redundancy or pattern is being compressed by these codes?
\item Can you go as far as identifying the probability distribution that these codes are (near-)optimal for?
\item Does the paper describe a proof-of-concept, a prototype implementation, or a ready-to-use software package?
\item What compression ratio is claimed (if any)? Are theoretical limits discussed?
\item Does the paper describe benchmarks? If so, what? Are other programs compared, or any standard metrics established?
\item If the codec is statistical, or otherwise involves parameterization from a training set, what corpus was used to optimize it?
\item Do the papers cite each other? Do they discuss similarities or differences between each other? Can you make any additional comparisons between the papers?
\item Can you think of any redundancies or patterns in the datasets-to-be-compressed that would be missed by these approaches?
\item Are there any other notable aspects or results of the papers?
\end{enumerate}
�

\end{document}
