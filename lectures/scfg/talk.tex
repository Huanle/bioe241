\documentclass{beamer}

\usepackage{url}

\newcommand\vs{\vspace{\baselineskip}}

\newcommand\titlecite{
{\em Of all natural systems, living matter preserves inscribed in its organization the largest amount of its own past history .... no other system is better}
 aufgehoben:
{\em constantly abolished and simultaneously preserved.}
\cite{PaulingZuckerkandl63}}

\newcommand\presentation[1]{ \section*{PRESENTATION: #1} }



% workaround for beamer bug
\providecommand\thispdfpagelabel[1]{}  % workaround

% presentation
\mode<presentation>
{
  \usetheme{Warsaw}
  % or ...

  \setbeamercovered{transparent}
  % or whatever (possibly just delete it)
}


\usepackage[english]{babel}
% or whatever

\usepackage[latin1]{inputenc}
% or whatever

\usepackage{times}
\usepackage[T1]{fontenc}
% Or whatever. Note that the encoding and the font should match. If T1
% does not look nice, try deleting the line with the fontenc.


\title[SCFGs] % (optional, use only with long paper titles)
{Stochastic Grammars}

\subtitle
{Stochastic Context-Free Grammars} % (optional)

\author% [Holmes] (optional, use only with lots of authors)
{I.~Holmes} % \inst{1} \and S.~Another\inst{2}
% - Use the \inst{?} command only if the authors have different
%   affiliation.

\institute[University of California, Berkeley] % (optional, but mostly needed)
{
%  \inst{1}%
  Department of Bioengineering\\
  University of California, Berkeley}
% - Use the \inst command only if there are several affiliations.
% - Keep it simple, no one is interested in your street address.

\date%[Short Occasion] % (optional)
{Spring semester}

\subject{Talks}
% This is only inserted into the PDF information catalog. Can be left
% out. 



% If you have a file called "university-logo-filename.xxx", where xxx
% is a graphic format that can be processed by latex or pdflatex,
% resp., then you can add a logo as follows:

% \pgfdeclareimage[height=0.5cm]{university-logo}{university-logo-filename}
% \logo{\pgfuseimage{university-logo}}



% Delete this, if you do not want the table of contents to pop up at
% the beginning of each subsection:
\AtBeginSubsection[]
{
  \begin{frame}<beamer>{Outline}
    \tableofcontents[currentsection,currentsubsection]
  \end{frame}
}


% If you wish to uncover everything in a step-wise fashion, uncomment
% the following command: 

%\beamerdefaultoverlayspecification{<+->}


\begin{document}

\begin{frame}
  \titlepage
\end{frame}

\begin{frame}{Outline}
  \tableofcontents
  % You might wish to add the option [pausesections]
\end{frame}



\section{Overview of transformational grammars}

\begin{frame}{}

\itemb
\item Overview: HMM profiles, HMM genefinders, SCFGs for RNA, repeats, beta-sheets; Natural Language Processing
\item What is a transformational grammar? Formal definition: terminals $\Omega$, nonterminals $\Phi$, transformation rules
 \itemb
 \item ``Language'' = set of strings generated by the grammar
 \item ``Parser'' = computer program to {\bf decide} if a given input string is in the language (returns ``true'' or ``false'')
 \item More generally, we're interested in parsers that compute scores (energies, probabilities) for a given input string
 \item These scores are associated with the transformation rules. The grammar is said to be score-{\em attributed} (Knuth)
 \iteme
\iteme
\end{frame}
\begin{frame}{}
\itemb
\item The Chomsky hierarchy of grammars and their associated parsers
 \itemb
 \item Regular grammars: finite-state machines (HMMs)
 \item Context-free grammars: pushdown automata (SCFGs)
 \item Context-sensitive grammars: finite-tape Turing machines (linear-bounded automata)
 \item Unrestricted grammars: infinite-tape Turing machines
 \iteme
\iteme
\end{frame}
\begin{frame}{}
\itemb
\item The Chomsky hierarchy of grammars and their associated parsers (lower levels)
 \itemb
 \item Regular grammars: finite-state machines (HMMs)
 \item Context-free grammars: pushdown automata (SCFGs)
  \itemb
  \item The {\bf parse tree}; the {\bf inside sequence} and {\bf outside sequence}
  \item Chomsky Normal Form; Eddy {\em et al}'s ``RNA Normal Form''
  \item Transformations to \& consequent universality of Chomsky Normal Form
  \item History:
Panini's ``Ashtadhyayi'': a CFG for Sanskrit (3959 rules (sutras), circa.300-700BC).
Bishop Robert Lowth's ``A Short Introduction to English Grammar'' (1762).
Chomsky's theory of universal generative grammars (1956 and onwards).
  \iteme
 \iteme
\iteme
\end{frame}
\begin{frame}{}
\itemb
\item The Chomsky hierarchy of grammars and their associated parsers (upper levels)
 \itemb
 \item Context-sensitive grammars: Turing machines with bounded tape = linear-bounded automata
  \itemb
  \item Big category: low-complexity sequence repeats, tandem repeats, other bounded correlations e.g. pseudoknots
  \item Lempel-Ziv compression algorithm (allowing local stutter) can be viewed as one of these
  \item Tree-Adjoining Grammars, Linear-Indexed Grammars, etc.
  \iteme
 \item Unrestricted grammars: complete Turing machines
 \iteme
\iteme
\end{frame}
\begin{frame}{}
\itemb
\item Summary of Chomsky hierarchy (NB regular $\subset$ context-free $\subset$ context-sensitive $\subset$ unrestricted)
\\
\small
\begin{tabular}{lllllllll}
Class & Example rule & Automaton & Complexity & Related models & Max-product & \multicolumn{2}{l}{Sum-product} & EM \\
\hline
Regular & $S \to x\ S$ & Finite-state & $O(L)$ & HMM, GHMM & Viterbi & Forward & Backward & Baum-Welch \\
Context-free & $S \to T\ U$ & Pushdown & $O(L^3)$ & SCFG & CYK & Inside & Outside & (Inside-Outside) \\
Context-sensitive & $S\ T \to T S$ & Linear-bounded & $O(|\Phi|^L)$ & TAG, RNRG & - & - & - & - \\
Unrestricted & $S\ T \to U$ & Turing machine & Undecidable & - & - & - & - & -
\end{tabular}
\normalsize
Here $S,T,U$ are nonterminals and $x$ is a terminal.
``Complexity'' refers to the time complexity of parsing a sequence of length $L$.
\iteme

\end{frame}

\section{RNA structure}

\begin{frame}{}

\itemb
\item Why RNA is important in evolution and cell biology
 \inone{RNA world; pre- and post-transcriptional regulation; Crick's idea of studying simple examples; ribotechnology}
\iteme
\end{frame}

\begin{frame}{}

\itemb
\item RNA structure terminology: basepairs, stems, loops, pseudoknots, kissing loops
\iteme
\end{frame}

\begin{frame}{}

Terms contributing to the free energy of a folded RNA structure ({\tt scor.berkeley.edu})
 \itemb
 \item Hydrogen bonding between bases. Canonical, noncanonical pairs
 \item Stacking energies due to overlap of $\pi$-orbitals of adjacent planar basepairs
  \inone{H-bonding and stacking terms can be combined and measured by direct experiment.}
 \item Unusual configurations: tetraloops, triloops, triple-A platforms
  \inone{Finite number of cases, so also amenable to experimental measurement.}
\iteme
\end{frame}

\begin{frame}{}
 \itemb
 \item Entropic cost of closing loops
  \itemb
  \item Theory: rods and Gaussian springs. Integrate out displacements, get likelihood ratio (Doi-Edwards, Isambert)
  \item Statistics of random walk, $\expect{|\Delta {\bf x}|^2} \propto t$, and self-avoiding walk, $\expect{|\Delta {\bf x}|^2} \propto t^{1+\epsilon}$
   \inone{Renormalisation (Edwards, de Gennes)}
  \item Empirical scaling laws fit experimental measurements
  \iteme
 \item Ligands: solvation, metal ions, small molecules
  \inone{General rules not yet known. Specific small-molecule binding requires conserved motifs (riboswitches).}
 \item Unlike proteins (where amino acid sidechains make multiple contacts),
a {\bf ``basepair stacking + convex loop penalty''} picture of the free energy seems reasonably accurate as a first approximation
(Zuker, Turner, Mathews...)
 \iteme
\end{frame}

\begin{frame}{}
\itemb
\item The Nussinov algorithm: Finds {\em strictly nested} foldback structures, i.e. excluding pseudoknots.
 \itemb
 \item RNA sequence $X$, length $L$, nucleotides $x_1 \ldots x_L$
 \item Let $H(x,x')=1$ if $xx'$ is a canonical Watson-Crick basepair, and $0$ otherwise
 \item Nussinov recursion finds the structure for $x_i \ldots x_j$ that has the most strictly nested canonical basepairs
\[
S(i,j) = \max \left( \begin{array}{c} S(i+1,j), \\ S(i,j-1), \\ S(i+1,j-1) + H(x_i,x_j), \\ \max_{i \leq k \leq j} (S(i,k) + S(k,j)) \end{array} \right)
\]
Best structure for $X$ is found by traceback from $S(1,L)$
 \iteme
\iteme
\end{frame}

\begin{frame}{}
Nussinov algorithm = parser for {\bf score-attributed grammar}

\begin{tabular}{rll|l}
\multicolumn{3}{c|}{Rule} & Score \\
\hline
$S$ & $\to$ & $S\ x$     & 0 \\
    &   $|$ & $x\ S$     & 0 \\
    &   $|$ & $x\ S\ x'$ & H(x,x') \\
    &   $|$ & $S\ S$     & 0 \\
    &   $|$ & $\epsilon$ & 0
\end{tabular}

If we allow $H(x,x')$ to be the free energy of basepair formation for $xx'$
(and assume a zero energy cost for any structural feature except basepairs),
then this becomes an {\bf energy-attributed grammar}.
Same recursion for $S$ now calculates ``ground state'' energy
for a very simple energy model ignoring all but hydrogen bonding terms for strictly nested basepairs.
\end{frame}

\begin{frame}{}
The partition function for this highly simplified model is
\begin{eqnarray*}
Z(i,j) & = & Z(i+1,j) + Z(i,j-1) + Z(i+1,j-1) \exp (-\beta H(x_i,x_j)) \\
& & + \sum_{i \leq k \leq j} Z(i,k) Z(k,j)
\end{eqnarray*}
where $\beta = 1/kT$ is an ``inverse temperature''.
NB: have simply changed $H \to \exp(-\beta H)$, $+ \to \times$ and $\max \to +$ in equation for $S$
\end{frame}

\begin{frame}{}
We can also have {\bf probability-attributed grammars} or {\bf stochastic grammars}.
Relationship between scoring schemes:
\\
\begin{tabular}{l|r}
Score & Form \\
\hline
Bayes     & $P(x)$ \\
Shannon   & $h(x) = -\log_2 P(x)$ \\
Boltzmann & $E(x) \simeq -\log_e P(x)$
\end{tabular}
\\
Specifically the Boltzmann probability uses a scaling factor (inverse temperature) and a partition function,
$P(x) = \frac{1}{Z} \exp[-\beta E(x)]$,
where $Z = \sum_x \exp[-\beta E(x)]$.
% \item Sankoff's $k$-loops
\end{frame}
\begin{frame}{}
Energy-attributed grammar (Zuker, MFOLD).
Parse tree for subsequence $X_i \ldots X_j$ must be rooted at $W$ or $V_{X_i X_j}$.
\\
\tiny
\begin{tabular}{rll|l}
LHS & & RHS & Energy \\
\hline
$W$ & $\to$ & $W\ x$         & 0 \\
    &   $|$ & $x\ W$         & 0 \\
    &   $|$ & $x\ V_{xy}\ y$ & $\alpha(x,y)$ \\
    &   $|$ & $W\ W$         & 0 \\
    &   $|$ & $x\ y$         & $\alpha(x,y)$ \\
    &   $|$ & $\epsilon$     & 0 \\
$V_{ab}$ & $\to$ & $z^n$                               & $h(n)$ \\
         &   $|$ & $x\ V_{xy}\ y$                      & $\alpha_S(x,y|a,b)$ \\
         &   $|$ & $x\ V_{xy}\ y\ z^n$                 & $\alpha(x,y) + b(n)$ \\
         &   $|$ & $z^n\ x\ V_{xy}\ y$                 & $\alpha(x,y) + b(n)$ \\
         &   $|$ & $z^m\ x\ V_{xy}\ y\ z^n$            & $\alpha(x,y) + i(m+n)$ \\
         &   $|$ & $x\ V_{xy}\ y\ W\ x'\ V_{x'y'}\ y'$ & $\alpha(x,y) + \alpha(x',y')$ \\
         &   $|$ & $\epsilon$                          & 0
\end{tabular}
\normalsize
\\
Here $x,y,x',y',z_i$ are terminals (nucleotides), and $z^n$ is shorthand for an $n$-nucleotide emission $z_1 \ldots z_n$.
\end{frame}
\begin{frame}{}
The scoring scheme is as follows:
\\
\small
\begin{tabular}{rl}
Free energy term & Meaning \\
\hline
$\alpha(x,y)$ & Energy of basepair $xy$ (end of stem, no stacking) \\
$\alpha_S(x,y|a,b)$ & Energy of basepair $xy$ stacked on top of basepair $ab$ \\
$h(n)$ & Hairpin loop enclosing $n$ bases \\
$b(n)$ & Asymmetric bulge of $n$ bases \\
$i(n)$ & Interior loop (symmetric bulge) with total of $n$ bases
\end{tabular}
\normalsize
\\
This scoring scheme was historically formulated in terms of Sankoff's ``$k$-loop decomposition''.
\\
NB Zuker's algorithm (grammar), like Nussinov's, excludes pseudoknots.
\\
Strictly, Zuker described the algorithm that finds the lowest-energy parse using the above grammar (CYK).
McCaskill described the algorithm for calculating the partition function (Inside)
and thus the posterior probabilities of individual basepairs (Outside).

\end{frame}

\section{Dynamic programming algorithms for SCFGs}

\begin{frame}{}

\itemb
\item Chomsky normal form. (RNA normal form is more useful in practise, but CNF is easier to present.) \\
For nonterminals $A,B,C \in \Phi$ and terminals $a \in \Omega$:
\begin{tabular}{rll|l}
\multicolumn{3}{c|}{Rule} & Name \\
\hline
$A$ & $\to$ & $B\ C$      & Bifurcation \\
    &   $|$ & $a$         & Emission \\
    &   $|$ & $\epsilon$  & Termination
\end{tabular} \\
Probabilities denoted by $P(\mbox{rule})$, e.g. $P(A \to BC)$
\iteme
\end{frame}
\begin{frame}{}
\itemb
\item Inside algorithm. \\
Let $I_A(i,k) = P(x_i \ldots x_{i+k} | A)$ be sum of probabilities for parse trees rooted in $A$ generating sequence $x_i \ldots x_{i+k}$.
\begin{eqnarray*}
I_A(i,k) & = & \left( \sum_B \sum_C \sum_{j=0}^k P(A \to BC) I_B(i,j) I_C(i+j,k-j) \right)
\\ & &
+ \left\{ \begin{array}{ll}
0 & \mbox{if $k>1$} \\
P(A \to x_i) & \mbox{if $k=1$} \\
P(A \to \epsilon) & \mbox{if $k=0$}
\end{array} \right\}
\end{eqnarray*}
NB loopy dependencies, e.g. if $P(A \to AA) \neq 0$ and $P(A \to \epsilon) \neq 0$.
It's common to try to avoid these when designing the grammar.
\iteme
\end{frame}
\begin{frame}{}
\itemb
\item Cocke-Younger-Kasami (CYK) algorithm. \\
Let $Y_A(i,k)$ be probability of ML parse tree rooted in $A$ generating sequence $x_i \ldots x_{i+k}$.
\begin{eqnarray*}
\lefteqn{Y_A(i,k) = \max} & & \\
& & \left( \begin{array}{c} \max_B \max_C \max_{j=0}^k P(A \to BC) Y_B(i,j) Y_C(i+j,k-j), \\
\left\{ \begin{array}{ll}
0 & \mbox{if $k>1$} \\
P(A \to x_i) & \mbox{if $k=1$} \\
P(A \to \epsilon) & \mbox{if $k=0$}
\end{array} \right\}
\end{array} \right)
\end{eqnarray*}
NB similar to Nussinov.
\iteme
\end{frame}
\begin{frame}{}
\itemb
\item Outside algorithm. \\
Let $O_A(i,k) = P(x_1 \ldots x_{i-1},A,x_{i+k+1} \ldots x_L|S)$
be sum of probabilities for incomplete parse trees rooted in $S$, ending in $A$ and generating outside sequence $x_1 \ldots x_{i-1}$ and $x_{i+k+1} \ldots x_L$.
\begin{eqnarray*}
O_A(i,k) & = & \sum_B \sum_C \left(
\sum_{j=0}^i O_C(i-j,j+k) P(C \to BA) I_B(i-j,j) \right. \\ & & \left. +
\sum_{j=0}^{L-i-k} O_C(i,j+k) P(C \to AB) I_B(i+k,j)
\right)
\end{eqnarray*}
with boundary condition $O_A(0,L) = \delta(A=S)$.
\iteme
\end{frame}
\begin{frame}{}
\itemb
\item Inside-Outside and posterior probabilities. \\
Posterior probability that parse tree contains a subtree with inside sequence $x_i \ldots x_{i+k}$ rooted in state $A$:
\[
P(A_{i,i+k}|X) = \frac{O_A(i,k) I_A(i,k)}{I_S(0,L)}
\]
Posterior probability of bifurcation $A_{i,j+k} \to B_{i,j} C_{i+j,k}$:
\small
\[
P(A_{i,j+k} \to B_{i,j} C_{i+j,k}|X) = \frac{O_A(i,j+k) P(A \to BC) I_B(i,j) I_C(i+j,k)}{I_S(0,L)}
\]
\normalsize
etc.
\iteme
\end{frame}
\begin{frame}{}
\itemb
\item Parameter estimation: we can write the parse tree likelihood as $\prod_i \theta_i^{n_i}$,
where $\theta_i$ is the probability of rule $i$ and $n_i$ is the number of times it was applied.
As with HMMs, the EM algorithm for SCFGs thus involves computing posterior expectations $\expect{n_i}$ for these counts,
and setting $\theta_i \propto \expect{n_i}$.
The posterior expectations are computed using the Inside-Outside algorithm, as shown for rules of the form $P(A \to BC)$.
\item There is a KYC-like analogue to Outside, that can be used to find ML parse tree including a particular subsequence.
\item Implementation issues: time complexity is $O(|\Phi|^3 L^3)$, memory is $O(|\Phi| L^2)$.
 \inone{Cubic factors in time complexity arise due to bifurcations, so minimize number of bifurcations for max efficiency (c.f. RNA normal form).}
\iteme

\end{frame}

\section{Beyond SCFGs}

\begin{frame}{Pair SCFGs, evolutionary SCFGs and tree transducers}

\itemb
\item Evolutionary SCFGs: PFOLD (xfold, evofold, etc.)
 \itemb
 \item As with Evolutionary HMMs, we can let the terminals be alignment columns
 \item Again, terminal emission likelihood $P(A \to a)$ is implemented as Felsenstein pruning
 \iteme
\iteme
\end{frame}
\begin{frame}{}
\itemb
\item Pair SCFGs: Evoldoer. Version of TKF that describes evolution of RNA secondary structure (Holmes 2005).
 \itemb
 \item Two kinds of TKF91 links model, recursively nested in a tree
 \item Stem sequences rooted in $S$ nonterminals; basepair alphabet $\Omega^2$; ends in an $L$
 \item Loop sequences rooted in $L$ nonterminals; nucleotide alphabet $\Omega$; $S$'s also allowed in sequence
  \inone{Really need to adapt TKF91 model to allow deletion prob to depend on symbol, otherwise $S$ substructures get deleted at same rate as nucleotides}
 \iteme
\item Pair SCFGs (e.g. Stemloc, QRNA). Heuristic, but with lots of go-faster stripes (pre-aligning \& pre-folding sequences).
\item Rules for composing Pair SCFGs exist (c.f. string transducers; can view conditionally normalized pair SCFGs as ``tree transducers'').
\iteme

\end{frame}

\begin{frame}{Graph grammars}

\itemb
\item Tree-adjoining grammars
 \itemb
 \item Aravind Joshi (1975, 1985)
 \iteme
\item Rivas-Eddy papers
 \itemb
 \item A dynamic programming algorithm for RNA structure prediction including pseudoknots. JMB 1999.
 \item The language of RNA: a formal grammar that includes pseudoknots. Bioinformatics 2000.
 \iteme
\item Graph grammars: easy to describe and simulate, attractive for biology; but how does their DP work?
\item Other grammars whose marginals are easy to compute by sum-product DP, e.g.
 \itemb
 \item stochastic tree grammars (Abe and Mamitsuka, ISMB 1994)
 \item context-sensitive HMMs with finite memory of last $N$ emitted characters (Yoon and Vaidyanathan, 2004)
 \iteme
\iteme

\end{frame}

\begin{frame}{Discriminative grammars}

Discriminative grammars: conditional log-linear models
\itemb
\item Recall HMMs and linear CRFs form a ``generative-discriminative pair''
\item The analogous discriminative model for SCFGs is a Conditional Log-Linear Model
\item This allows a great deal of physics-like parameterization (loop entropies, stacking free energies, terminal mismatch...) without having to introduce many new nonterminals
 \inone{Do, Woods and Batzoglou. ``CONTRAfold: RNA secondary structure prediction without physics-based models.'' Bioinformatics 22:14, pp e90-e98}
\iteme

\end{frame}



\section*{Summary}

\begin{frame}{Summary}

  % Keep the summary *very short*.
  \begin{itemize}
  \item SCFGs
  \end{itemize}

\end{frame}


\end{document}
