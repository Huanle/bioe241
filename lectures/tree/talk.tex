\documentclass{beamer}

\usepackage{url}

\newcommand\vs{\vspace{\baselineskip}}

\newcommand\titlecite{
{\em Of all natural systems, living matter preserves inscribed in its organization the largest amount of its own past history .... no other system is better}
 aufgehoben:
{\em constantly abolished and simultaneously preserved.}
\cite{PaulingZuckerkandl63}}

\newcommand\presentation[1]{ \section*{PRESENTATION: #1} }



% workaround for beamer bug
\providecommand\thispdfpagelabel[1]{}  % workaround

% presentation
\mode<presentation>
{
  \usetheme{Warsaw}
  % or ...

  \setbeamercovered{transparent}
  % or whatever (possibly just delete it)
}


\usepackage[english]{babel}
% or whatever

\usepackage[latin1]{inputenc}
% or whatever

\usepackage{times}
\usepackage[T1]{fontenc}
% Or whatever. Note that the encoding and the font should match. If T1
% does not look nice, try deleting the line with the fontenc.


\title[Trees] % (optional, use only with long paper titles)
{Phylogenetic trees}

\subtitle
{Likelihood computations and inference} % (optional)

\author% [Holmes] (optional, use only with lots of authors)
{I.~Holmes} % \inst{1} \and S.~Another\inst{2}
% - Use the \inst{?} command only if the authors have different
%   affiliation.

\institute[University of California, Berkeley] % (optional, but mostly needed)
{
%  \inst{1}%
  Department of Bioengineering\\
  University of California, Berkeley}
% - Use the \inst command only if there are several affiliations.
% - Keep it simple, no one is interested in your street address.

\date%[Short Occasion] % (optional)
{Spring semester}

\subject{Talks}
% This is only inserted into the PDF information catalog. Can be left
% out. 



% If you have a file called "university-logo-filename.xxx", where xxx
% is a graphic format that can be processed by latex or pdflatex,
% resp., then you can add a logo as follows:

% \pgfdeclareimage[height=0.5cm]{university-logo}{university-logo-filename}
% \logo{\pgfuseimage{university-logo}}



% Delete this, if you do not want the table of contents to pop up at
% the beginning of each subsection:
\AtBeginSubsection[]
{
  \begin{frame}<beamer>{Outline}
    \tableofcontents[currentsection,currentsubsection]
  \end{frame}
}


% If you wish to uncover everything in a step-wise fashion, uncomment
% the following command: 

%\beamerdefaultoverlayspecification{<+->}


\begin{document}

\begin{frame}
  \titlepage
\end{frame}

\begin{frame}{Outline}
  \tableofcontents
  % You might wish to add the option [pausesections]
\end{frame}


\section{Likelihoods}

\newcommand\rootpi{\pi_{x_1}}
\newcommand\mxx[2]{M(t_{#1#2})_{x_{#1} x_{#2}}}
\newcommand\mxy[2]{M(t_{#1#2})_{x_{#1} y_{#2}}}
\newcommand\xydelta[1]{\delta_{x_{#1} y_{#1}}}

\begin{frame}{Tree Notation}

%\item Examples of why phylogeny is important
% \inone{mammalian evolution, virus evolution, orthology}
Notation and model: let...
\vspace{\baselineskip}

\begin{tabular}{ll}
 $x_n$ & be (actual) state of node $n$ \\
 $y_n$ & be (observed) character at leaf node $n$ \\
 $X = \{x_n\}$ & be the set of all node states \\
 $Y = \{ y_n \}$ & be the set of all observed characters at leaf nodes \\
 $Y_n$ & be the set of all $y_m$ descended from node $n$ \\
 $\overline{Y_n} = Y \setminus Y_n$ & be the set of all remaining $y_m$ \\
 $t_{mn}$ & be evolutionary ``distance'' (time) from $m$ to $n$
\end{tabular}

\vspace{\baselineskip}
Root node is node 1.

For convenience, number nodes in \alert{preorder} (parents < kids)

\end{frame}

\begin{frame}{Conditional \& Prior Probabilities}

Parent $x_p$, child $x_c$:
\[
P(x_c|x_p) = \mxx{p}{c} = \exp({\bf R}t_{pc})_{x_p x_c}
\]

Leaf node $c$:
\[
P(y_c|x_c) = \delta_{x_c y_c}
\]
i.e. observation $y_c$ fully specifies state $x_c$.
We can drop the $x_c$:

\[
P(y_c|x_p) = \sum_{x_c} P(x_c|x_p) P(y_c|x_c) = \mxy{p}{c}
\]

Assumption: Ur-ancestor was in evolutionary equilibrium
\[
P(x_1) = \pi_{x_1}
\]

(Root node is always node 1)

\end{frame}


\begin{frame}{Felsenstein's Algorithm: Two Taxa}

\itemb
\item Felsenstein's pruning algorithm: our first DP
 \itemb
 \item Consider tree $T_1$:
\Tree [.1 2 3 ]
%\begin{parsetree}
% ( .1. .2. .3. )
%\end{parsetree}
with the following dependencies between state variables $X$ and observations $Y$:
%\begin{parsetree}
% ( .$x_1$. ( .$x_2$. .$y_2$. ) ( .$x_3$. .$y_3$. )  )
%\end{parsetree}
\Tree [ .$x_1$ [ .$x_2$ $y_2$ ] [ .$x_3$ $y_3$ ] ]
  \itemb
  \item Joint likelihood of all nodes:
\[
P(X,Y) = P(x_1) P(x_2|x_1) P(x_3|x_1) P(y_2|x_2) P(y_3|x_3)
\]
  \item Marginal likelihood of leaves:
\begin{eqnarray*}
P(Y) & = & \sum_X P(X,Y) \\
& = & \sum_{x_1} P(x_1) P(y_2|x_1) P(y_3|x_1)
\end{eqnarray*}
  \iteme
 \iteme
\iteme

% \Tree [.S This [.VP [.V is ] \qroof{a simple tree}.NP ] ]

\end{frame}

\begin{frame}{Felsenstein's Algorithm: Four Taxa}

\small
Now consider tree $T_2$:
%\begin{parsetree}
% ( .1. ( .2. .4. .5. ) ( .3. .6. .7. ) )
%\end{parsetree}
\Tree [ .1 [ .2 4 5 ] [ .3 6 7 ] ]

\vspace{\baselineskip}

Joint likelihood of all nodes:
\begin{eqnarray*}
\lefteqn{P(X,Y)} \\
& = & P(x_1) P(x_2|x_1) P(x_3|x_1) P(x_4|x_2) P(x_5|x_2) P(x_6|x_3) P(x_7|x_3) \prod_{n=4}^7 P(y_n|x_n)
\end{eqnarray*}

\end{frame}

\begin{frame}{Felsenstein's Algorithm: Rearranging Terms}

Tree $T_2$:
\Tree [ .1 [ .2 4 5 ] [ .3 6 7 ] ]
and model
\Tree [ .$x_1$ [ .$x_2$ $y_4$ $y_5$ ] [ .$x_3$ $y_6$ $y_7$ ] ]

Marginal likelihood of leaves:
\begin{eqnarray*}
\lefteqn{P(Y) = \sum_X P(X,Y)} \\
& \scriptstyle = & \scriptstyle \sum_{x_1} \sum_{x_2} \sum_{x_3} P(x_1) P(x_2|x_1) P(x_3|x_1) P(y_4|x_2) P(y_5|x_2) P(y_6|x_3) P(y_7|x_3) \\
& \scriptstyle = & \scriptstyle \sum_{x_1} P(x_1) \left[ \left( \sum_{x_2} P(x_2|x_1) \left[ P(y_4|x_2) P(y_5|x_2) \right] \right) \left( \sum_{x_3} P(x_3|x_1) \left[ P(y_6|x_3) P(y_7|x_3) \right] \right) \right]
\end{eqnarray*}

Terms in square brackets have the form:
\[
F_n(x_n) = P(\{y_i:i\ \mbox{descended from}\ n\}|x_n) = P(Y_n|x_n)
\]

\end{frame}

\begin{frame}{Felsenstein's Algorithm: Generalization}

\itemb
  \item Let $C_n$ be the set of immediate children of $n$ \\ (so e.g. $C_1 = \{ 2,3 \}$)
  \item Rule to compute $F_n$ is then
\begin{eqnarray*}
F_n(x_n) & = & \left\{ \begin{array}{ll} \displaystyle
\prod_{\mbox{$c \in C_n$}} \left( \sum_{x_c} P(x_c|x_n) F_c(x_c) \right) & \mbox{if $n$ is internal} \\
P(y_n|x_n) & \mbox{if $n$ is a leaf}
\end{array} \right. \\
P(Y) & = & \sum_{x_1} P(x_1) F_1(x_1)
\end{eqnarray*}
  \item This is the \alert{pruning} algorithm (Felsenstein, 1981).
\iteme

\end{frame}

\begin{frame}{Felsenstein's Algorithm: Recursion}

\begin{eqnarray*}
F_n(x_n) & = & \left\{ \begin{array}{ll} \displaystyle
\prod_{\mbox{$c \in C_n$}} \left( \sum_{x_c} P(x_c|x_n) F_c(x_c) \right) & \mbox{if $n$ is internal} \\
P(y_n|x_n) & \mbox{if $n$ is a leaf}
\end{array} \right. \\
P(Y) & = & \sum_{x_1} P(x_1) F_1(x_1)
\end{eqnarray*}

\itemb
  \item Term after ``$\prod$'' sign in $F_n$ can be written $E_c(x_p) = P(Y_c|x_p) = \sum_{x_c} P(x_c|x_p) F_c(x_c)$ \\ where $p$ is parent of $c$
  \item An instance of the \alert{sum-product algorithm} on a factor graph
\iteme

\end{frame}

\begin{frame}{Pulley Principle}

\itemb
\item ``Pulley principle'' for reversible models. % Consider tree $T_1$ again:
\Tree [ .1 2 3 ]
%\begin{parsetree}
% ( .1. .2. .3. )
%\end{parsetree}
 \itemb
 \item Using $\pi_i M(t)_{ij} = \pi_j M(t)_{ji}$ and ${\bf M}(t){\bf M}(t')={\bf M}(t+t')$
\begin{eqnarray*}
P(y_2,y_3) & = & \sum_{x_1} \pi_{x_1} M(t_{12})_{x_1 y_2} M(t_{13})_{x_1 y_3} \\
& = & \sum_{x_1} \pi_{x_2} M(t_{12})_{y_2 x_1} M(t_{13})_{x_1 y_3} \\
& = & \pi_{y_2} M(t_{12}+t_{13})_{y_2 y_3} \\
& = & \pi_{y_3} M(t_{12}+t_{13})_{y_3 y_2}
\end{eqnarray*}
 \item $P(y_2,y_3)$ depends only on $t_{12}+t_{13}$
 \item Can slide root node (like a pulley) w/out affecting likelihood 
 \item Corollorary: can re-root tree at any node, including any leaf
 \iteme
\iteme

\end{frame}

\begin{frame}{Alignment Probability}

\itemb

\item So far we have looked at $P(C|T)$, the probability of an individual alignment column, conditioned on a tree
\item Probability of an entire alignment conditioned on tree, $P(A|T)$, is product of column probabilities:
\[
P(A|T) = \prod_{C \in A} P(C|T)
\]
 \inone{Denote maximum likelihood tree by $T_{\mbox{\tiny ML}} = \argmax_T P(A|T)$}
\item Note that so far we have neglected the indel history that is also (partially) specified by the alignment
\iteme

\end{frame}

\section{Posterior probabilities}

\begin{frame}{Ancestral State Reconstruction on a Phylogeny}

\itemb
\item Motivation:
 \itemb
 \item Probability distribution of ancestral state, $P(x_n|Y)$
 \item Probability distribution of $p \to c$ branch, $P(x_p,x_c|Y)$
 \iteme
\item Recall tree notation:
 \itemb
 \item $Y = \{ y_i \}$ is the set of all leaf states
 \item $Y_n$ contains all $y_i$ descended from node $n$
 \item $\overline{Y_n}$ contains all $y_i$ \alert{not} descended from node $n$
 \inone{Note $Y_1 \equiv Y$, since node 1 is the root node.}
 \iteme
\iteme

\end{frame}

\begin{frame}{Probabilities of Ancestral States}

\itemb
 \item Felsenstein's pruning algorithm computes $F_n(x_n) = P(Y_n|x_n)$
  \inone{and thus $P(Y) = \sum_{x_1} \pi(x_1) F_1(x_1)$}
 \item We will now give recursions for $G_n(x_n) = P(x_n,\overline{Y_n})$.
 \item With these we can easily get posterior probabilities for...
  \itemb
  \item State of ancestral node
  \item State of ancestral branch
  \iteme
\iteme

\end{frame}

\begin{frame}{State of Ancestral Node}

\begin{eqnarray*}
F_n(x_n) & = & P(Y_n|x_n) \\
& & \\
G_n(x_n) & = & P(x_n,\overline{Y_n}) \\
& & \\
P(x_n|Y) & = & \frac{P(x_n,Y)}{P(Y)} \\
& = & \frac{1}{P(Y)} P(x_n,\overline{Y_n}) P(Y_n|x_n) \\
& = & \frac{1}{P(Y)} G_n(x_n) F_n(x_n)
\end{eqnarray*}

\end{frame}

\begin{frame}{State of Ancestral Branch}

\itemb
  \item Joint state of ancestral branch
% (parent $p$, child $c$, sibling $s$):
\Tree [ .$p$ $c$ $s$ ]
\begin{eqnarray*}
P(x_p,x_c|Y) & = & \frac{P(x_p,x_c,Y)}{P(Y)} \\
& = & \frac{1}{P(Y)} P(x_p,\overline{Y_p}) P(x_c|x_p) P(Y_c|x_c) P(Y_s|x_p) \\
& = & \frac{1}{P(Y)} G_p(x_p) P(x_c|x_p) F_c(x_c) E_s(x_p)
\end{eqnarray*}
where $E_s(x_p) = \sum_{x_c} P(x_c|x_p) F_s(x_s)$ as before
\iteme

\end{frame}

\begin{frame}{Peeling Recursion}

\itemb
 \item Recursion for $G_c(x_c) = P(x_c,\overline{Y_c})$
\[
G_c(x_c) = \left\{ \begin{array}{ll}
\sum_{x_p} G_p(x_p) P(x_c|x_p) E_s(x_p) & \mbox{if $c>1$ (not root)} \\
P(x_c) & \mbox{if $c=1$ (root)}
\end{array} \right.
\]
 \inone{aka Elston-Stewart peeling algorithm}
\iteme

\end{frame}

\begin{frame}{Degenerate characters}

\small
\begin{tabular}{|r|l|}
\hline
IUPAC ambiguity code & Possibilities \\
\hline
R & A G \\
Y & C T \\
M & A C \\
K & G T \\
S & C G \\
W & A T \\
H & A C T \\
B & C G T \\
V & A C G \\
D & A G T \\
N or X & A C G T \\
\hline
\end{tabular}

\normalsize
\itemb
 \item Recall the hidden state $x_i$ behind observation $y_i$
 \item $P(y_i|x_i)=A_{x_i y_i}$ reflects ambiguous $y_i$
 \item Can be extended to read quality scores, etc.
 \item $P(X,Y)$ is product of $\pi_i$'s, $M_{ij}$'s and $A_{ij}$'s
\iteme

\end{frame}

\section{Factor graphs}


\newcommand\PXYFactorGraph{
\Tree [ .$f_1$ [ .$x_1$
 [ .$f_2$ [ .$x_2$ [ .$f_4$ [ .$x_4$ $f_{4a}$ ] ] [ .$f_5$ [ .$x_5$ $f_{5a}$ ] ] ] ]
 [ .$f_3$ [ .$x_3$ [ .$f_6$ [ .$x_6$ $f_{6a}$ ] ] [ .$f_7$ [ .$x_7$ $f_{7a}$ ] ] ] ]
] ]
}

\begin{frame}{Bigraph representation of $P(X,Y)$}

2-coloring: variables $X = \{ x_i \}$, functions ${\cal F} = \{ f_n \}$

\vspace{.5\baselineskip}

\begin{tabular}{p{.35\textwidth}p{.65\textwidth}}
\PXYFactorGraph
&
\tiny
\begin{eqnarray*}
f_1(x_1) & = & \pi_{x_1} \\
f_2(x_1,x_2) & = & \mxx{1}{2} \\
f_3(x_1,x_3) & = & \mxx{1}{3} \\
f_4(x_2,x_4) & = & \mxx{2}{4} \\
f_5(x_2,x_5) & = & \mxx{2}{5} \\
f_6(x_3,x_6) & = & \mxx{3}{6} \\
f_7(x_3,x_7) & = & \mxx{3}{7} \\
f_{4a}(x_4)  & = & A_{x_4 y_4} \\
f_{5a}(x_5)  & = & A_{x_5 y_5} \\
f_{6a}(x_6)  & = & A_{x_6 y_6} \\
f_{7a}(x_7)  & = & A_{x_7 y_7} \\
P(x_1 \ldots x_7;y_4 \ldots y_7) & = & f_1(x_1) f_2(x_1,x_2) f_3(x_1,x_3) \\
& & \times f_4(x_2,x_4) f_5(x_2,x_5) f_6(x_3,x_6) f_7(x_3,x_7) \\
& & \times f_{4a}(x_4) f_{5a}(x_5) f_{6a}(x_6) f_{7a}(x_7) \\
& = & g(x_1 \ldots x_7)
\end{eqnarray*}
\end{tabular}

\end{frame}

\begin{frame}{Bigraph representation of $P(X,Y)$ showing functions}

\Tree [ .$\pi_{x_1}$ [ .$x_1$
 [ .$\mxx{1}{2}$ [ .$x_2$ [ .$\mxx{2}{4}$ [ .$x_4$ $A_{x_4 y_4}$ ] ] [ .$\mxx{2}{5}$ [ .$x_5$ $A_{x_5 y_5}$ ] ] ] ]
 [ .$\mxx{1}{3}$ [ .$x_3$ [ .$\mxx{3}{6}$ [ .$x_6$ $A_{x_6 y_6}$ ] ] [ .$\mxx{3}{7}$ [ .$x_7$ $A_{x_7 y_7}$ ] ] ] ]
] ]

\end{frame}


\newcommand\PYExprTree{
\tiny
\Tree [.$\sum_{x_1}$ [.$\otimes$ $f_1$ [ .$F_1:\otimes\quad\quad$
 [ .$E_2:\sum_{x_2}\quad$ [ .$\otimes$ $f_2$ [ .$F_2:\otimes\quad\quad$ [ .$E_4:\sum_{x_4}\quad$ [ .$F_4:\otimes\quad\quad$ $f_4$ $f_{4a}$ ] ] [ .$E_5:\sum_{x_5}\quad$ [ .$F_5:\otimes\quad\quad$ $f_5$ $f_{5a}$ ] ] ] ] ]
 [ .$E_3:\sum_{x_3}\quad$ [ .$\otimes$ $f_3$ [ .$F_3:\otimes\quad\quad$ [ .$E_6:\sum_{x_6}\quad$ [ .$F_6:\otimes\quad\quad$ $f_6$ $f_{6a}$ ] ] [ .$E_7:\sum_{x_7}\quad$ [ .$F_7:\otimes\quad\quad$ $f_7$ $f_{7a}$ ] ] ] ] ]
] ] ]
}

\begin{frame}{Expression tree of $P(Y)$}

\PYExprTree

\end{frame}


\newcommand\PYandPXY{
\begin{frame}{Expression tree of $P(Y)$ and factor graph of $P(X,Y)$}

\noindent\begin{tabular}{p{.7\textwidth}p{.3\textwidth}}
\makebox{\PYExprTree} & \makebox{\tiny\PXYFactorGraph}
\end{tabular}

\end{frame}
}

\PYandPXY

\begin{frame}{Factor Graphs}
\itemb
 \item The idea of computing marginals and posteriors on a graphical network of variables has been generalized
  \itemb
  \item ``\alert{Factor graphs}'', ``Graphical models'', ``Bayesian networks'', ``Markov random fields''
  \item ``\alert{Sum-product algorithm}'', ``Message-passing'', ``Belief propagation''
  \iteme
 \item Here we use the Factor Graph variant of the formalism
  \inone{Kschichang, Frey \& Loeliger, IEEE 47:2, February 2001.}
\iteme
\end{frame}

\begin{frame}{The summary operator}
\itemb
 \item The ``summary'' or ``not-sum'' operator, $\notsum{} f(\cdot)$
  \itemb
  \item Suppose $V \subseteq W \subseteq X$.
  \item The \alert{summary of $f(W)$ w.r.t. $V$}
is obtained by summing $f(W)$ over all $x_i \notin V$.
The result is a function of $V$.
  \inone{For example...}
\[
\notsum{x_3,x_4} f(x_2,x_3,x_4,x_5,x_6) = \sum_{x_2} \sum_{x_5} \sum_{x_6} f(x_2,x_3,x_4,x_5,x_6)
\]
  \item Define the summary of $f$ w.r.t. its own arguments as $f$ itself:
\[
\notsum{W}f(W) \equiv f(W)
\]
  \item \alert{The summaries of a joint likelihood are the marginals:} \\ If $g(X)=P(X)$, then $\notsum{V} g(V) = P(V)$
  \iteme
\iteme
\end{frame}

\begin{frame}{Message-passing}
\itemb
 \item Factor graphs contain
  \itemb
  \item \alert{function nodes} ($f_i$)
  \item \alert{variable nodes} ($x_i$)
  \item An edge $f_i$---$x_j$ indicates $x_j$ is a parameter of $f_i$
  \iteme
 \item Computation proceeds by \alert{message-passing}
 \item A ``message'' is actually a pre-computed function \\ over some subset of the variables in the graph
 \item The message from node $a$ to node $b$ is written $\msg{a}{b}$
\iteme
\end{frame}

\begin{frame}{Message-passing}
\itemb
  \item Message from $f_i$ to $x_j$:
\[
\msg{f_i}{x_j} = \notsum{x_j} f_i \prod_{k \neq j} \msg{x_k}{f_i}
\]
where $\{ \msg{x_k}{f_i} \}$ are the incoming messages to $f_i$
  \item Message from $x_i$ to $f_j$:
\[
\msg{x_i}{f_j} = \prod_{k \neq j} \msg{f_k}{x_i}
\]
where $\{ \msg{f_k}{x_i} \}$ are the incoming messages to $x_i$
\iteme
\end{frame}

\begin{frame}{Message-passing}

\begin{eqnarray*}
\msg{f_i}{x_j} & = & \notsum{x_j} f_i \prod_{k \neq j} \msg{x_k}{f_i} \\
\msg{x_i}{f_j} & = & \prod_{k \neq j} \msg{f_k}{x_i}
\end{eqnarray*}

 The message sent from a node $n$ on an edge $e$ is...
   \itemb
   \item the product of:
    \enumb
    \item the local function at $n$ \\
{\small (or the unit function, $x_n \to 1$, if $n$ is a variable node);}
    \item all messages received at $n$ on edges {\em other} than $e$,
    \enume
   \item summarized for the variable associated with $e$.
   \iteme
\end{frame}


\begin{frame}{Phylogenetic message-passing}

 For a phylogenetic tree, the messages are as follows \\ ($c-p-s$ denoting child---parent---sibling)
  \itemb
  \item Leaves-to-root:
\begin{eqnarray*}
\msg{f_{na}}{x_n} & = & f_{na}(x_n) \\
\msg{x_n}{f_n} & = & F_n(x_n) \\
\msg{f_c}{x_p} & = & E_c(x_p)
\end{eqnarray*}
  \item Root-to-leaves: 
\begin{eqnarray*}
\msg{f_n}{x_n} & = & G_n(x_n) \\
\msg{x_p}{f_c} & = & G_p(x_p) E_s(x_p)
\end{eqnarray*}
  \iteme
\end{frame}

\PYandPXY


\begin{frame}{Factor Graphs}
\itemb
\item Many applications in biology...
 \itemb
 \item Markov models
 \item Phylogenetic trees
 \item Hidden Markov models
 \item Inference on ontologies
 \item Markov random fields (images, gene networks, ...)
 \iteme
\item ...and beyond...
 \itemb
 \item Error-correcting codes

 \iteme
\iteme
\end{frame}


\begin{frame}{Changing the Operators}
\itemb
\item The ``sum'' and ``product'' operators in the sum-product algorithm are arbitrary
 \itemb
 \item The sum-product algorithm works by factorising terms like
\[
\sum_{x_1} \sum_{x_2} f(x_1)f(x_2) = \left( \sum_{x_1} f(x_1) \right) \left( \sum_{x_2} f(x_2) \right)
\]
 \item This in turn relies on the distributive law,
\[
a \times (b+c)=a \times b + a \times c
\]
and can be applied to any semiring with operators $(\otimes,\oplus)$
  \iteme
\iteme
\end{frame}

\begin{frame}{Changing the Operators: ML inference}
\itemb
 \item For example, consider
\begin{eqnarray*}
a \oplus b & \equiv & \max(a,b) \\
a \otimes b & \equiv & ab
\end{eqnarray*}
 \itemb
 \item The pruning algorithm now calculates $\max_X P(X,Y)$
 \item It is straightforward to recover $\argmax_X P(X,Y)$
  \itemb
   \item i.e. the MAP (\alert{Maximum {\em A Posteriori}}) ancestral state history
   \item note that $\argmax_X P(X,Y) = \argmax_X P(X|Y)$
   \item this may not give the same results as $\argmax_{x_i} P(x_i|Y)$
  \iteme
 \iteme
\iteme
\end{frame}

\begin{frame}{Changing the Operators: Parsimony}
\itemb
 \item Another example: consider 
\begin{eqnarray*}
a \oplus b & \equiv & \min(a,b) \\
a \otimes b & \equiv & a + b
\end{eqnarray*}
and set $M(t)_{ij} = A_{ij} = 1-\delta_{ij}$ and $\pi_i=0$
\inone{Then $M$ counts the number of substitutions, and pruning returns the \alert{most parsimonious} imputation of $\{ x_i \}$}
\item A number of neat results for parsimony exist
\inone{e.g. a branch-and-bound algorithm to find the most parsimonious tree}
\item but parsimony is v restrictive ``model''; strong assumptions:
 \itemb
 \item all substitutions equally likely, no back-substitutions...
 \item parsimony not really compatible w/ probabilistic methods
 \iteme
\iteme
\end{frame}

% \begin{frame}{Expression tree of $P(Y)$ showing functions}
% 
% \tiny
% \Tree [.$\sum_{x_1}$ [.$\otimes$ $\pi_{x_1}$ [ .$\otimes$
% [ .$\sum_{x_2}$ [ .$\otimes$ $\mxx{1}{2}$ [ .$\otimes$ [ .$\sum_{x_4}$ [ .$\otimes$ $\mxx{2}{4}$ $A_{x_4 y_4}$ ] ] [ .$\sum_{x_5}$ [ .$\otimes$ $\mxx{2}{5}$ $A_{x_5 y_5}$ ] ] ] ] ]
% [ .$\sum_{x_3}$ [ .$\otimes$ $\mxx{1}{3}$ [ .$\otimes$ [ .$\sum_{x_6}$ [ .$\otimes$ $\mxx{3}{6}$ $A_{x_6 y_6}$ ] ] [ .$\sum_{x_7}$ [ .$\otimes$ $\mxx{3}{7}$ $A_{x_7 y_7}$ ] ] ] ] ]
% ] ] ]
% 
% \end{frame}

\section*{Summary}

\begin{frame}{Summary}

  % Keep the summary *very short*.
  \begin{itemize}
  \item Message-passing on phylogenetic trees
   \itemb
   \item Felsenstein's pruning algorithm
   \item Elston-Stewart peeling algorithm
   \iteme
  \item Pulley principle \& reversibility
  \end{itemize}

\end{frame}

\end{document}
