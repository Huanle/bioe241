\chapter{Homology search}

Goal of this lab:
Use HMMer \cite{FinnClementsEddy2011} and other tools to build and use a homology profile from a family of homologous protein domain sequences.

\section{Download and install DART}

This lab does not use the DART software, but many of the other labs do,
and the DART software contains some tools for working with the Stockholm-format files that HMMer also uses.

Download and install DART from \url{http://biowiki.org/DownloadingDart}

The tools for manipulating Stockholm format are described at \url{http://biowiki.org/StockholmTools}
and located in {\tt dart/perl}.

\section{Download and install MUSCLE}

MUSCLE \cite{Edgar2004b} is a fast multiple alignment program.
Source code, and binaries for several platforms, are provided.
Download it from \url{http://www.drive5.com/muscle/}

\section{Download and install HMMer}

Download the HMMer source code from \url{http://hmmer.janelia.org/}

Configure and install the software as per the instructions in the User Guide.

\section{Follow the tutorial in the HMMer user guide}

The HMMer distribution contains example files for a simple tutorial (chapter 3 of the User Guide).
The examples include several protein domains: globins, fibronectin type III domains, the {\em Drosophila} Sevenless protein, and protein kinases.

Work through this tutorial.
Ensure that you are able to complete all the examples.

\section{Use external tools to manipulate the HMMer files}

A few exercises to go slightly beyond the HMMer tutorial:
\begin{itemize}
\item Convert the Stockholm-format HMMer example alignments into FASTA format, using the {\tt dart/perl/stockholm2fasta.pl} tool.
Try to generate both gapped and ungapped FASTA format (hint: use the {\tt -h} option to {\tt dart/perl/stockholm2fasta.pl} to list other command-line options)
\item Use MUSCLE to build an alignment from the ungapped FASTA-format sequences, then convert to Stockholm using {\tt dart/perl/fasta2stockholm.pl}
\end{itemize}
